\begin{abstract}
Unmanned systems in precision agriculture increasingly rely on wireless communication to close the sensing-communication-control (SCC) loop. In practice, wireless links are often bandwidth-limited and intermittent, rendering periodic state transmission inefficient and overly conservative, particularly over finite mission horizons.

This paper investigates a fundamental performance--communication trade-off in closed-loop unmanned systems subject to intermittent state updates. We consider a discrete-time linear system controlled by a fixed linear quadratic regulator (LQR), in which state information is delivered through an error-based event-triggered communication mechanism. Intermittent communication induces a grow-and-reset prediction-error process at the controller side, providing a transparent system-level interpretation of how communication decisions shape closed-loop dynamics. From this perspective, event-triggered communication naturally emerges as a communication-regulated endogenous input to an otherwise nominally stable system.

For this setting, we establish a communication-regulated finite-time input-to-state stability (FT-ISS) characterization under bounded disturbances, in which the closed-loop state bound holds uniformly over the mission horizon and scales explicitly with the triggering threshold. This result reveals a tunable trade-off between closed-loop performance and communication usage. Simulations motivated by precision agriculture scenarios confirm that substantial communication reductions can be achieved with only moderate performance degradation.
\end{abstract}




\begin{IEEEkeywords}
event-triggered control, communication constraints, unmanned systems, networked control
\end{IEEEkeywords}

\section{Introduction}

In apple cultivation, timely pest and disease detection is critical to crop yield and quality. For large-scale orchards, unmanned aerial vehicles (UAVs) equipped with high-resolution and multispectral sensors are commonly deployed to collect lesion and pest information, which is transmitted via wireless links to ground stations for decision-making and control, thereby closing the sensing-communication-control (SCC) loop. However, orchard environments are characterized by complex terrain, foliage occlusion, and long operational distances, leading to bandwidth-limited and intermittent wireless communication.

At the same time, UAV endurance is constrained, and periodic transmission of high-rate sensory data can be redundant, energy-intensive, and detrimental to coverage efficiency. Moreover, pest and disease symptoms typically evolve gradually relative to the sensing rate, rendering periodic state transmission overly conservative. These considerations motivate communication strategies that adapt transmission timing to the system state, rather than relying on fixed sampling schedules.

Unmanned systems operating over large spatial scales increasingly rely on wireless communication to close the sensing-control loop, a setting that has been extensively studied in the networked control systems literature~\cite{hespanha2007survey,schenato2007foundations}. In precision agriculture, UAV-based platforms are widely adopted for monitoring and actuation; however, their wireless links are inherently bandwidth-limited, intermittent, and energy-constrained~\cite{zeng2016wireless}. Meanwhile, the underlying plant and vehicle dynamics in such applications are often smooth and slowly varying, making periodic state transmission unnecessarily conservative.

Event-triggered control has therefore emerged as a promising alternative, in which communication decisions are adapted to the system state rather than fixed sampling schedules~\cite{tabuada2007event,heemels2012introduction}. In such architectures, intermittent communication induces a mismatch between the true system state and the controller-side prediction. Importantly, this mismatch is not an exogenous disturbance imposed by the environment, but an endogenous quantity actively regulated through communication events that reset the prediction error when it exceeds a prescribed threshold.

Most existing results on event-triggered control focus on the design of triggering mechanisms and on establishing asymptotic stability, ultimate boundedness, or Zeno-freeness under intermittent communication. While these analyses are powerful, they often obscure the system-level role of communication in shaping closed-loop dynamics. In particular, communication typically appears as a technical condition embedded in Lyapunov arguments, rather than as an explicit component whose effect on system behavior can be directly interpreted and tuned.

Moreover, many mission-oriented unmanned systems, including agricultural UAVs, operate over prescribed finite time horizons dictated by battery limits, coverage requirements, and task scheduling. In such settings, performance guarantees over a finite mission duration are often more relevant than asymptotic behavior as time tends to infinity. This motivates analytical frameworks that explicitly characterize how intermittent communication influences closed-loop performance over finite horizons.

In this paper, we adopt a system-level viewpoint and interpret event-triggered communication as a communication-regulated mechanism acting on the controller-side prediction error. Intermittent transmissions induce a grow-and-reset error process: in the absence of communication, the prediction error evolves according to the plant dynamics, while communication events actively contract the error through estimator updates. From this perspective, communication-induced estimation errors can be treated as an endogenous input acting on an otherwise nominally stable closed-loop system, naturally leading to a finite-time input-to-state stability (FT-ISS) interpretation.

To isolate this performance-communication trade-off in a transparent and analytically tractable manner, we consider a baseline yet practically relevant setting: a discrete-time linear time-invariant system controlled by a fixed linear quadratic regulator (LQR), subject to bounded disturbances. State measurements are delivered to the controller through an error-based event-triggered communication policy parameterized by a single triggering threshold. Rather than designing new triggering mechanisms or controllers, our objective is to characterize how intermittent communication shapes closed-loop behavior and how the triggering threshold serves as a unified SCC design parameter. This perspective aligns with recent efforts on SCC co-design in networked systems~\cite{zhang2019networked}.

The contributions of this paper are summarized as follows:
\begin{itemize}
\item We show that intermittent communication can be interpreted as a grow-and-reset mechanism acting on the controller-side prediction error, yielding a transparent system-level description of communication effects on closed-loop dynamics.
\item We establish a communication-regulated FT-ISS result under bounded disturbances, explicitly revealing how the closed-loop state bound scales with the triggering threshold over a finite mission horizon.
\item Through simulations motivated by precision agriculture scenarios, we demonstrate clear Pareto-type trade-offs between closed-loop performance and communication usage, as well as robustness under practical communication impairments.
\end{itemize}

\section{Related Work}

Event-triggered control and networked control systems have been extensively studied, with a rich body of results on stability analysis, performance guarantees, and communication reduction; see, e.g.,~\cite{tabuada2007event,heemels2012introduction,ge2020dynamic}. Existing works primarily focus on triggering mechanism design and on establishing stability, Zeno-freeness, or robustness under network-induced imperfections such as packet loss and denial-of-service attacks~\cite{dolk2016event,feng2020networked,zhao2020dynamic}.

Beyond model-based designs, data-driven approaches have been proposed to synthesize event-triggered controllers directly from data while preserving stability guarantees~\cite{de2019formulas,de2023event,wang2023data}. Extensions to distributed, stochastic, and large-scale settings have also been explored, where communication constraints interact with uncertainty and scalability.

While these works provide powerful tools for reducing communication and ensuring stability, communication typically appears as an implicit technical condition within Lyapunov-based analyses rather than as an explicit system-level component shaping closed-loop behavior. As a result, it is often difficult to directly interpret how communication decisions regulate estimation accuracy and closed-loop performance.

In contrast, this paper adopts a system-level perspective that models intermittent communication as a grow-and-reset mechanism acting on the controller-side prediction error. By treating communication-induced estimation errors as an endogenous input to an otherwise nominally stable system, we obtain a transparent and analytically tractable characterization of communication effects.

Moreover, most existing results emphasize asymptotic stability, whereas mission-oriented systems often operate over prescribed finite horizons. This motivates finite-horizon stability characterizations that explicitly capture how bounded disturbances and communication-induced errors jointly affect system evolution. Leveraging input-to-state stability (ISS) concepts~\cite{jiang2001input,sontag2015iss} and their finite-time counterparts, we formulate a communication-regulated FT-ISS bound whose dependence on the triggering threshold is explicit.


\begin{figure*}[t]
    \centering
    \includegraphics[
        width=0.95\linewidth,
        trim= 20 210 20 280,
        clip,
    ]{figs/ssc_framework.png}
    \caption{Ground-station-centric SCC framework with event-triggered communication. State measurements collected by the UAV are transmitted intermittently to a remote ground station according to an error-based triggering rule. Between communication events, the ground station propagates a model-based state prediction, while successful transmissions reset the estimation error through an observer update. This grow-and-reset prediction-error process induces a communication-regulated endogenous input to an otherwise nominally stable closed-loop system. Control actions are generated by a fixed LQR controller, closing the feedback loop over a finite mission horizon.}

    \label{fig:scc_framework}
\end{figure*}

\section{System Model and Communication Architecture}

\subsection{Finite-Horizon Mission Setup}

We consider an unmanned aerial vehicle (UAV) executing a task over a prescribed finite mission horizon
\[
k \in \{0,1,\dots,T\}, \quad T < \infty,
\]
such as crop monitoring or precision spraying in agricultural environments. The finite horizon reflects practical mission constraints arising from battery limitations, coverage requirements, and task scheduling.

All state estimation, control computation, and communication scheduling are performed at a remote ground station. The UAV acts as a sensing and actuation unit, transmitting measurements to the ground station over a bandwidth-limited wireless uplink and applying control inputs received via a downlink. This architecture is representative of many practical SCC deployments in precision agriculture.

\subsection{Plant Dynamics with Bounded Disturbances}

The UAV dynamics are modeled as a discrete-time linear time-invariant (LTI) system
\begin{equation}
x_{k+1} = A x_k + B u_k + w_k,
\label{eq:lti}
\end{equation}
where $x_k \in \mathbb{R}^n$ denotes the system state and $u_k \in \mathbb{R}^m$ is the control input. The disturbance term $w_k$ captures unmodeled dynamics, environmental effects such as wind, and discretization errors.

We assume that the process disturbance is bounded over the mission horizon:
\begin{equation}
\|w_k\|_2 \le \bar w, \quad \forall k \in \{0,\dots,T\}.
\label{eq:w_bound}
\end{equation}

State measurements collected onboard the UAV are given by
\begin{equation}
y_k = C x_k + v_k,
\label{eq:meas}
\end{equation}
where $v_k$ represents measurement noise due to sensor imperfections and quantization. The measurement noise is also assumed to be bounded:
\begin{equation}
\|v_k\|_2 \le \bar v, \quad \forall k \in \{0,\dots,T\}.
\label{eq:v_bound}
\end{equation}

These bounded disturbance assumptions are standard in networked and event-triggered control and are consistent with the FT-ISS analysis developed later.

\subsection{Remote State Prediction and Event-Triggered Updates}

Due to communication constraints, state measurements are not transmitted at every time step. Instead, the ground station maintains a remote state prediction
\begin{equation}
\hat x_k^- = A \hat x_{k-1} + B u_{k-1},
\label{eq:predict}
\end{equation}
which evolves according to the nominal plant dynamics and previously applied control inputs.

Let $\gamma_k \in \{0,1\}$ denote the communication indicator at time $k$, where $\gamma_k = 1$ indicates that a measurement update is successfully received by the ground station. Upon reception, the remote estimate is updated as
\begin{equation}
\hat x_k =
\begin{cases}
\hat x_k^- + L (y_k - C \hat x_k^-), & \gamma_k = 1,\\
\hat x_k^-, & \gamma_k = 0,
\end{cases}
\label{eq:estimator}
\end{equation}
where $L$ is a fixed observer gain.

Define the controller-side prediction error
\begin{equation}
\tilde x_k \triangleq x_k - \hat x_k.
\label{eq:tilde}
\end{equation}
This error captures the mismatch between the true system state and the state information available to the controller and plays a central role in the subsequent analysis.

Combining \eqref{eq:lti}-\eqref{eq:estimator}, the prediction error evolves according to the switched dynamics
\begin{equation}
\tilde x_k =
\begin{cases}
(I-LC)\big(A\tilde x_{k-1}+w_{k-1}\big) - L v_k, & \gamma_k = 1,\\
A\tilde x_{k-1}+w_{k-1}, & \gamma_k = 0,
\end{cases}
\label{eq:error}
\end{equation}
which exhibits a grow-and-contract structure: in the absence of communication, the error propagates according to the open-loop dynamics, while measurement updates actively contract the error through the observer correction.

\subsection{Event-Triggered Uplink Communication}

Uplink transmissions are governed by an innovation-based event-triggered communication policy
\begin{equation}
\gamma_k =
\mathbf{1}\!\left\{
\|y_k - C\hat x_k^-\|_2 > \delta
\right\},
\label{eq:trigger}
\end{equation}
where $\delta > 0$ is a design threshold regulating the trade-off between communication frequency and estimation accuracy.

This triggering rule transmits measurements only when the innovation exceeds a prescribed tolerance, thereby avoiding unnecessary communication when the controller-side prediction remains sufficiently accurate. The threshold $\delta$ serves as a key design parameter that couples sensing, communication, and control.

\subsection{Uniform Boundedness of the Prediction Error}

\begin{lemma}[Uniform Boundedness of Event-Triggered Prediction Error]
\label{lem:tilde_bound}
Suppose that the pair $(A,C)$ is detectable and that the observer gain $L$ is chosen such that $(I-LC)A$ is Schur stable. Under the triggering rule \eqref{eq:trigger} and bounded disturbances
\eqref{eq:w_bound}-\eqref{eq:v_bound}, there exist constants $\kappa_1>0$ and
$\kappa_2>0$, depending only on $(A,C,L)$, such that
\begin{equation}
\|\tilde x_k\|_2 \le \kappa_1 \delta + \kappa_2(\bar w + \bar v),
\qquad \forall k \in \{0,\dots,T\}.
\label{eq:tilde_uniform}
\end{equation}
\end{lemma}

\begin{proof}

When $\gamma_k = 0$, the triggering condition $\|y_k - C\hat{x}_k^{-}\|_2 \le \delta$
implies that the innovation remains bounded, preventing unbounded growth of the estimation error in the absence of communication. When $\gamma_k = 1$, the observer update induces a contraction through the stable map $(I-LC)A$.

Since $(I-LC)A$ is Schur stable and both process and measurement disturbances are uniformly bounded, the switched error dynamics admit a compact positively invariant set. Standard arguments from event-triggered observer analysis then imply that the size of this invariant set scales linearly with the triggering threshold $\delta$ and the disturbance bounds $\bar w$ and $\bar v$. Consequently, there exist constants $\kappa_1>0$ and $\kappa_2>0$ such that \eqref{10} holds for all $k\in\{0,\ldots,T\}$.

\end{proof}

\subsection{Remote Control Law and Communication-Induced Disturbance}

The control input is generated at the ground station using the remote state estimate:
\begin{equation}
u_k = -K \hat x_k,
\label{eq:control}
\end{equation}
where the feedback gain $K$ is chosen such that $A - BK$ is Schur stable.

Substituting \eqref{eq:control} into \eqref{eq:lti} yields the closed-loop dynamics
\begin{equation}
x_{k+1} = (A - BK)x_k + BK\tilde x_k + w_k,
\label{eq:cl}
\end{equation}
where the prediction error $\tilde x_k$ appears as a communication-induced endogenous disturbance acting on an otherwise nominally stable system. This representation forms the basis for the communication-regulated FT-ISS analysis in the next section.

\section{Communication-Regulated Finite-Time ISS}

This section formalizes the system-level interpretation introduced earlier by establishing a FT-ISS characterization of the closed-loop dynamics under event-triggered communication. The analysis explicitly reveals how communication-induced estimation errors enter the system as an endogenous disturbance whose magnitude is regulated by the triggering threshold.

\begin{theorem}[Communication-Regulated Finite-Time ISS]
\label{thm:ftiss}
Consider the closed-loop system
\begin{equation}
x_{k+1} = (A-BK)x_k + d_k,
\qquad
d_k \triangleq BK\tilde x_k + w_k,
\label{eq:cl_ftiss}
\end{equation}
evolving over the finite horizon $k \in \{0,\dots,T\}$.
Suppose that the feedback gain $K$ is chosen such that $A-BK$ is Schur stable and that the disturbances satisfy
\eqref{eq:w_bound}-\eqref{eq:v_bound}.
Then the system \eqref{eq:cl_ftiss} is finite-time input-to-state stable (FT-ISS). In particular, there exist constants $c_x>0$, $c_d>0$, and $\rho \in (0,1)$ such that
\begin{equation}
\|x_k\|_2
\le
c_x \rho^k \|x_0\|_2
+
c_d \max_{0\le i \le k-1}\|d_i\|_2,
\qquad \forall k \in \{0,\dots,T\}.
\label{eq:ftiss_std}
\end{equation}

Moreover, under the event-triggered estimation mechanism described in Section~IV,
\begin{equation}
\max_{0\le i \le k-1}\|d_i\|_2
\le
\|BK\|\big(\kappa_1\delta + \kappa_2(\bar w+\bar v)\big) + \bar w,
\label{eq:d_uniform}
\end{equation}
where $\kappa_1$ and $\kappa_2$ are the constants defined in Lemma~\ref{lem:tilde_bound}.
Consequently, the state trajectory satisfies the bound
\begin{equation}
\|x_k\|_2
\le
c_x \rho^k \|x_0\|_2
+
c_d\Big(
\|BK\|\big(\kappa_1\delta + \kappa_2(\bar w+\bar v)\big) + \bar w
\Big),
\qquad \forall k \in \{0,\dots,T\}.
\label{eq:ftiss_comm}
\end{equation}
\end{theorem}

\begin{proof}
Since $A-BK$ is Schur stable, there exists a positive definite matrix $P \succ 0$ such that
\[
(A-BK)^\top P (A-BK) - P = -Q
\]
for some $Q \succ 0$.
Consider the quadratic Lyapunov function $V(x) = x^\top P x$.
Standard Lyapunov arguments for discrete-time systems imply that
\eqref{eq:ftiss_std} holds for the system \eqref{eq:cl_ftiss} with input $d_k$ over the finite horizon.

The bound on $d_k$ follows directly from Lemma~\ref{lem:tilde_bound} and the disturbance assumptions
\eqref{eq:w_bound}-\eqref{eq:v_bound}. Substituting this bound into \eqref{eq:ftiss_std} yields
\eqref{eq:ftiss_comm}, completing the proof.
\end{proof}

\begin{remark}[Communication-Regulated Practical Stability]
The FT-ISS estimate \eqref{eq:ftiss_comm} implies finite-time practical stability of the closed-loop system over the mission horizon. In particular, the triggering threshold $\delta$ explicitly regulates the magnitude of the communication-induced disturbance and thus serves as a unified SCC design parameter. Increasing $\delta$ reduces uplink communication at the expense of a larger state bound, while decreasing $\delta$ improves closed-loop performance at the cost of more frequent transmissions.
\end{remark}

\section{Simulation Results}

This section presents numerical simulations that illustrate and validate the proposed communication-regulated SCC framework.
All results are obtained over a finite mission horizon using a fixed stabilizing controller, in direct correspondence with the theoretical setting
developed in the preceding sections.

\subsection{Simulation Setup}

We consider a planar UAV modeled as a discrete-time double integrator with sampling period $T_s = 0.1$~s. The state is
$x_k = [p_x, v_x, p_y, v_y]^\top$, where $(p_x,p_y)$ and $(v_x,v_y)$ denote position and velocity, respectively, and the control input is
$u_k = [a_x, a_y]^\top$. A LQR is designed offline under full-state availability, with weighting matrices
$Q = \mathrm{diag}(10, 1, 10, 1)$ and $R = 0.1 I$.
The resulting feedback gain $K$ renders $A-BK$ Schur stable and is kept fixed throughout all simulations.

The UAV executes a lawnmower-style coverage task over a $200 \times 100$~m field, representative of precision agriculture missions.
Simulations are conducted over a finite horizon $T$ and averaged over 30 Monte Carlo runs with randomized initial conditions.
Unless otherwise specified, bounded process disturbances are included to induce nontrivial prediction-error growth.
State estimation and communication scheduling are performed at a remote ground station using the event-triggered mechanism described in Section~IV.

\subsection{Prediction-Error Dynamics under Event-Triggered Communication}

\begin{figure}[t]
    \centering
    \includegraphics[width=0.95\linewidth]{figs/fig_B_time_response.pdf}
    \caption{Time-domain visualization of the grow-and-reset prediction-error mechanism induced by event-triggered communication.
    The top panel shows the evolution of the prediction error norm $\|\tilde x_k\|_2$ together with the triggering threshold $\delta$.
    Between communication events, the error grows according to the open-loop prediction dynamics; when the innovation exceeds $\delta$,
    a transmission is triggered and the observer update contracts the error.
    The bottom panel reports the cumulative number of successfully delivered uplink packets, exhibiting a staircase structure aligned with the error resets.}
    \label{fig:time_response}
\end{figure}

Figure~\ref{fig:time_response} illustrates the evolution of the controller-side prediction error under event-triggered communication for a representative triggering threshold $\delta$.
Between communication events, the controller relies on model-based prediction, and the error evolves according to the open-loop estimation dynamics.
When the prediction error exceeds the threshold, a communication event is triggered, resulting in an observer update that contracts the error.

Also shown is the cumulative number of transmitted uplink packets, which exhibits a staircase structure aligned with the error resets.
The prediction error remains confined within a threshold-dependent envelope and does not grow unbounded despite intermittent communication.
This grow-and-reset behavior provides a direct time-domain illustration of the boundedness result in Lemma~\ref{lem:tilde_bound}
and highlights the role of communication in actively regulating estimation accuracy.

\subsection{Performance-Communication Trade-Off}

\begin{figure}[t]
    \centering
    \includegraphics[width=0.95\linewidth]{figs/fig_A_pareto_tradeoff.pdf}
    \caption{Performance-communication trade-off induced by the triggering threshold $\delta$.
    The figure reports the quadratic control cost versus the total number of delivered uplink packets over the finite mission horizon.
    Event-triggered communication (ET) achieves a smooth Pareto-type frontier and consistently outperforms periodic (PER) and random (RAND)
    transmission under identical communication budgets.
    A distinct knee region highlights an operating point where substantial communication savings are achieved with only moderate performance degradation.}
    \label{fig:pareto}
\end{figure}

We next examine how the triggering threshold $\delta$ regulates the trade-off between closed-loop performance and communication usage.
Figure~\ref{fig:pareto} reports the quadratic control cost as a function of the total number of delivered uplink packets over the mission horizon.

Event-triggered communication achieves a smooth Pareto-type frontier and consistently outperforms periodic and random transmission strategies under identical communication budgets.
A distinct knee region is observed, indicating an operating regime in which substantial communication savings are achieved with only moderate performance degradation.
This behavior is consistent with the communication-regulated FT-ISS bound in Theorem~\ref{thm:ftiss}, which predicts that increasing $\delta$ enlarges the effective disturbance induced by intermittent communication while reducing communication usage.

\subsection{Comparison under Matched Communication Budgets}

\begin{figure}[t]
    \centering
    \includegraphics[width=0.95\linewidth]{figs/fig_C_budget_curves.pdf}
    \caption{Time-domain comparison between event-triggered (ET) and periodic (PER) communication under matched communication budgets.
    Both strategies deliver approximately the same number of uplink packets over the mission horizon.
    Despite identical communication usage, event-triggered communication yields smaller state deviations and improved transient performance
    by allocating communication resources in a state-dependent manner.}
    \label{fig:budget_comparison}
\end{figure}

While the Pareto frontier provides a global characterization of the performance-communication trade-off, a direct time-domain comparison further clarifies the structural advantage of state-dependent communication.
Figure~\ref{fig:budget_comparison} compares event-triggered and periodic communication strategies selected to deliver approximately the same number of uplink packets over the mission horizon.

Despite identical communication budgets, event-triggered communication yields smaller state deviations and improved transient performance.
This improvement arises from the state-dependent allocation of communication resources rather than from increased transmission frequency.
By prioritizing updates when the prediction error is large, event-triggered communication makes more effective use of limited uplink capacity.

\subsection{Robustness under Communication Imperfections}

\begin{figure}[t]
    \centering
    \includegraphics[width=0.95\linewidth]{figs/fig_D_robustness_panel.pdf}
    \caption{Robustness of event-triggered communication under practical communication impairments.
    The figure compares the normalized quadratic cost of event-triggered and periodic communication under measurement noise,
    finite-bit quantization, packet loss, and moderate model mismatch.
    All results are reported under matched communication budgets and normalized by a high-communication periodic baseline.
    Event-triggered communication consistently outperforms periodic transmission, while preserving the qualitative structure
    of the performance-communication trade-off.}
    \label{fig:robustness}
\end{figure}

Finally, the robustness of the proposed framework is assessed under practical communication impairments commonly encountered in agricultural deployments.
The simulations incorporate measurement noise, finite-bit quantization, bursty packet loss modeled by a Gilbert-Elliott channel, and moderate model mismatch.
These effects enter the closed-loop dynamics as bounded disturbances, consistent with the FT-ISS formulation.

Figure~\ref{fig:robustness} summarizes the resulting performance under these impairments by comparing event-triggered and periodic communication at matched communication budgets.
Across all scenarios, event-triggered communication consistently outperforms periodic transmission, and the qualitative structure of the performance-communication trade-off is preserved.
These results indicate that the benefits of event-triggered communication arise from its intrinsic state-dependent regulation of estimation error rather than from idealized communication assumptions.

\section{Conclusion}

This paper provided a system-level reinterpretation of event-triggered SCC architectures, in which intermittent communication is explicitly modeled as a communication-regulated
endogenous input acting on the closed-loop system over a finite mission horizon. Focusing on a linear system controlled by a fixed LQR, we interpreted event-triggered communication as a grow-and-reset mechanism acting on the controller-side prediction error, enabling a transparent characterization of how communication decisions shape closed-loop dynamics. An error-based event-triggered communication policy was analyzed, with the triggering threshold serving as an explicit SCC design parameter. A communication-regulated FT-ISS characterization was established under bounded disturbances, revealing explicit scaling of the closed-loop state bound with the triggering threshold. Simulations motivated by precision agriculture scenarios illustrated clear Pareto-type trade-offs between control performance and communication usage, confirming that substantial communication reductions can be achieved with only moderate performance degradation.

